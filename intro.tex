\begin{frame}{Introduction}
  \framesubtitle{The Process}
  \begin{itemize}
    \item Process steps:
    \begin{itemize}
      \item Strip is unwound from the right roll
      \item Strip is rolled between the middle pair of rolls
      \item Strip is wound up by the left roll
      \item All rolls are driven by DC motors
    \end{itemize}
    \item Process goal: control the output thickness
  \end{itemize}
\end{frame}

\begin{frame}{Introduction}
  \framesubtitle{Sensors and Actuators}
  \begin{itemize}
    \item Actuators: 3 DC motors, armature current controlled
    \item Sensors:
    \begin{itemize}
      \item 3 velocity sensors
      \item 2 traction sensors
      \item 2 thickness sensors
    \end{itemize}
    \item Current setup $\Rightarrow$ only control sheet traction
  \end{itemize}
\end{frame}

\begin{frame}{Introduction}
  \framesubtitle{Controller Architecture}
  \begin{itemize}
    \item Cascade plant \\$\Rightarrow$ Cascade controller:
    \begin{itemize}
      \item Inner loop: DC motor speed control
      \item Outer loop: traction control
    \end{itemize}
    \item Traction system has differential input \\$\Rightarrow$ "master--slave" architecture
    \begin{itemize}
    \item Master: steady speed setpoint \\$\Rightarrow$ zero static error, disturbance rejection
    \item Slave: small signal speed \\$\Rightarrow$ tracking
    \end{itemize}
  \end{itemize}
\end{frame}

\begin{frame}{In this Presentation}
\tableofcontents
\end{frame}
