
\begin{frame}{Thickness}
	\begin{itemize}
		\item Thickness control not technically feasible due to numbers of DAC outputs needed
		\item Design would more be difficult because
		\begin{itemize}
			\item Thickness -- traction -- velocity relation: non-linear, look up tables are needed
			\item Some bias point parameters must be currently set by hand
		\end{itemize}
\end{itemize}
\end{frame}

\begin{frame}{Thickness}
\framesubtitle{Sensors and actuators}
Actuators:
\begin{itemize}
	\item Left Motor
	\item Rolling Motor
	\item Right Motor
\end{itemize}
Sensors:
\begin{itemize}
\item Thickness sensor before/after the rolling
\item Traction sensor before/after the rolling
\item Velocity sensors for each of the motors
\item Rolling force sensor
\end{itemize}
\end{frame}

\begin{frame}{Thickness}
\framesubtitle{Controller considerations}
\begin{itemize}
\item Cascade control:
\begin{itemize}
	\item Left Motor - P controller
	\item Rolling Motor - PI controller - Master
	\item Right Motor - P controller
\end{itemize}
\item Possible to get away with master/slave structure for the traction as well?
\item Traction control using P-PI for master, probably just P for slave (what about the overshoot?)
\item Possible to only work with small signals, or should the look up table be used during operation as well?
\end{itemize}
\end{frame}

\begin{frame}
\begin{center}
\Huge Thank you!\\
Questions?
\end{center}
\end{frame}
